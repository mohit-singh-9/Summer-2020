\renewcommand{\theequation}{\theenumi}
\begin{enumerate}[label=\thesection.\arabic*.,ref=\thesection.\theenumi]
\numberwithin{equation}{enumi}

\item  In $\triangle OCD$ 
		\begin{align}
		OD = OC = r  \\
    	CD = r \\
		\end{align}
  
   
    $\therefore \triangle OCD $ is an equilateral triangle , 
    \begin{align}
    \angle COD = \beta = 60\degree
    \end{align}
  
\item  In $\triangle CBD$ \\
Using the Theorem : Angle subtended by chord at the centre of circle is twice the angle subtended by it at any other point on the circle, we get
\begin{align}
\angle CBD &= \frac{\angle COD}{2} \\ 
 &= \frac{60\degree}{2} \\
 &= 30\degree
\end{align}
\begin{align} 
 \implies \alpha &= 30\degree
 \label{eq:eq_4}
\end{align}
    
\item In $\triangle BCA$, \\
We know that, angle subtended by a diameter at any point on circle is 90\degree.
\begin{align}
\angle BCA &= 90\degree 
\end{align}
\begin{align}
\implies \angle BEC &= 90\degree
\label{eq:eq_5}
\end{align}
    
\item Applying the sum of interior angles in $\triangle EBC$ 
\begin{align}
\angle BCE + \alpha + \angle BEC = 180\degree
\label{eq:eq_6}
\end{align}
Using \eqref{eq:eq_4}, \eqref{eq:eq_5} and \eqref{eq:eq_6}, we get
\begin{align}
\angle BEC &= 60\degree
\end{align}
\begin{align}
\therefore \angle AEB = 60\degree 
\end{align}
    
Hence proved.
    
\end{enumerate}   
       