\noindent From Fig. \ref{fig: circle_latex}\\
    In $\triangle OCD$  \newline 
    OD = OC = r (Radius of circle) \\
    CD = r (Given)\\
   
    $\therefore \triangle OCD $ is an equilateral triangle , \newline
    So  $\angle COD $ = $\theta$ = 60\degree
    \\
    \\
  
    In $\triangle CBD$  \\
    $\angle CBD$ = $\alpha$ = $\frac{1}{2}$.$\angle COD$ = $\frac{1}{2}$. 60\degree = 30\degree \\
  Using the Theorem : Angle subtended by chord at the centre of circle is twice the angle subtended by it at any other point on the circle
    \\
    \\
    In $\triangle BCA$ \newline
    $\angle BCA$ = 90\degree   [ Angle subtended by a diameter at any point on circle is 90\degree]
    \newline
    So, $\angle BCE$ = 90\degree
    \newline
    \newline
    Now in $\triangle EBC$ \newline
    $\angle BCE$ + $\alpha$ + $\angle BEC$ = 180\degree  [Triangle sum property]\newline
    90\degree + 30\degree + $\angle BEC$ = 180\degree \newline
    $\angle BEC$ = 60\degree \newline
    
   $\therefore$ $\angle AEB$ = 60\degree \newline
   Hence proved.
    
    
       