\renewcommand{\theequation}{\theenumi}
\begin{enumerate}[label=\thesection.\arabic*.,ref=\thesection.\theenumi]
\numberwithin{equation}{enumi}

\item First we need to prove that angle subtended by a diameter at a point on circumference is 90\degree. \\
This can be proved if the scalar product of $\vec{BD}$ and $\vec{AD}$ is 0.
\begin{align}
\brak{\vec{BD}}.\brak{\vec{AD}} &= \brak{\vec{D}-\vec{B}}.\brak{\vec{D}-\vec{A}} \\
&= \norm{D}^2 - \vec{D}.\vec{A} - \vec{B}.\vec{D} + \vec{B}.\vec{A}
\end{align}
Since $\vec{A}$ and $\vec{B}$ are equal in magnitude but opposite in direction , $\vec{A}$ = - $\vec{B}$. Also $\norm{A}$= $\norm{B}$ = $\norm{D}$ = $r$ . On substituting,
\begin{align}
\brak{\vec{BD}}.\brak{\vec{AD}} &= 0
\end{align}
Hence $\vec{BD} \perp \vec{AD}$; 
\begin{align}
\angle BDA = \angle ADE = 90\degree
\label{eq:angleADE}
\end{align}

\item To find $\gamma$ ,
consider the equation
\begin{align}
\norm{AC}^2 &= \brak{\vec{AC}}. \brak{\vec{AC}} \\
&= \brak{\vec{C} - \vec{A}}.\brak{\vec{C} - \vec{A}} \\
&= \norm{C}^2 + \norm{A}^2 + 2.\norm{A}.\norm{C}.\cos\theta
\end{align}
On sustituting the values, we get 
\begin{align}
\norm{AC} &= 1.53
\label{eq:AC}
\end{align}
Similarly 
\begin{align}
\norm{AD}^2 &= \brak{\vec{AD}}. \brak{\vec{AD}} \\
&= \brak{\vec{D} - \vec{A}}.\brak{\vec{D} - \vec{A}} \\
&= \norm{D}^2 + \norm{A}^2 + 2.\norm{A}.\norm{D}.\cos\theta_1
\end{align}
On sustituting the values, we get 
\begin{align}
\norm{AD} &= 3.174
\label{eq:AD}
\end{align}
Consider the scalar product of $\vec{AC}$ and $\vec{AD}$,
\begin{align}
\brak{\vec{AC}}.\brak{\vec{AD}} &= \norm{AC}\norm{AD}\cos\gamma
\label{eq:gen}
\end{align}
For L.H.S of \eqref{eq:gen},
\begin{align}
\brak{\vec{AC}}.\brak{\vec{AD}} &= \brak{\vec{C} - \vec{A}}.\brak{\vec{D} - \vec{A}}\\
 &= \vec{C}.\vec{D} - \vec{C}.\vec{A} - \vec{A}.\vec{D} + \norm{A}^2 \\
 &= 4\cos\beta - 4\cos\theta - 4\cos\theta_1 + 4
\end{align}
On substituting the angles , we get
\begin{align}
\brak{\vec{AC}}.\brak{\vec{AD}} &= 4.207
\label{eq: LHS}
\end{align}

Now substituting values from \eqref{eq: LHS}, \eqref{eq:AD}, \eqref{eq:AC} in \eqref{eq:gen}, we get $\gamma$.
\begin{align}
\cos\gamma = 0.866 \\
\gamma= 30\degree
\label{eq: gamma}
\end{align}



    
\item Applying the sum of interior angles in $\triangle ADE$ 
\begin{align}
\angle ADE + \gamma + \angle AED = 180\degree
\label{eq:eq_6}
\end{align}
Using \eqref{eq:angleADE}, \eqref{eq: gamma}, we get
\begin{align}
\angle AED &= 60\degree
\end{align}
\begin{align}
\therefore \angle AEB = 60\degree 
\end{align}
    
Hence proved.
    
\end{enumerate}   
       