\renewcommand{\theequation}{\theenumi}
\begin{enumerate}[label=\thesection.\arabic*.,ref=\thesection.\theenumi]
\numberwithin{equation}{enumi}
\begin{figure}[!ht]
\centering
\includegraphics[width= \columnwidth]{./figs/circle/circle.eps}
\end{figure}

\begin{table}[!ht]
\centering
\input{./tables/circle/inp.tex}
\caption{Input Table for construction}
\label{table:table1}	
\end{table}
\begin{table}[!ht]
\centering
\input{./tables/circle/der.tex}
\caption{Derived values while construction}
\label{table:table2}	
\end{table}
\item Let $\vec{O} = \myvec{x_1\\y_1}$ be the centre of the circle and $r$ be the radius of the circle. Since centre lies on the line, it satisfies the line equation
\begin{align}
\myvec{1&-3}\vec{O} = 11
\end{align}
\begin{align}
x_1 - 3y_1 = 11 
\label{eq:line1}
\end{align}

\item Also the circle passes through $\myvec{2\\3} and \myvec{-1\\1}$.Let these points be $\vec{P}$ and $\vec{Q}$ repectively. So the distance between centre and these points will be equal to the radius.
\begin{align}
\norm{\vec{P} - \vec{O}} = \norm{\vec{Q}- \vec{O}} = r 
\end{align}
On solving we get the equation
\begin{align}
6x_1 + 4y_1 = 11
\label{eq:line2}
\end{align}

\item The line equations from \eqref{eq:line1} and \eqref{eq:line2}, can be solved to get $\vec{O}$.
\begin{align}
\myvec{1&-3\\6&4}\myvec{x_1\\y_1} &= \myvec{11\\11} \\
\myvec{x_1\\y_1} &= \myvec{1&-3\\6&4}^{-1}\myvec{11\\11} \\
\myvec{x_1\\y_1} &= \frac{1}{22}\myvec{77\\-55}
\end{align}
Hence $\vec{O}$ = $\myvec{\frac{7}{2}\\\frac{-5}{2}}$

\item Sustituting $\vec{O}$ we get $r$= 5.7
\item Equation of circle is 
\begin{align}
\norm{\vec{x}- \vec{O}}= 5.7
\end{align}
\item The python code for the figure
\begin{lstlisting}
codes/circle/circle.py
\end{lstlisting}


\end{enumerate}