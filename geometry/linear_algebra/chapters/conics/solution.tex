\renewcommand{\theequation}{\theenumi}
\begin{enumerate}[label=\thesection.\arabic*.,ref=\thesection.\theenumi]
\numberwithin{equation}{enumi}

\item A conic section has the following equation 
\begin{align}
Ax^2+Bxy+Cy^2+Dx+Ey+F &= 0
\label{eq:alg_conic}
\end{align}
The equation is expressed in vector form is as follows
\begin{align}
\label{eq:vec_conic}
\vec{x}^T \myvec{A & B/2\\B/2 & C}\vec{x}+ \myvec{D & E}\vec{x}+F &= 0
\end{align}

\begin{enumerate}

\item 
$2x^2 - 7x + 3 = 0$ can be expressed as 
\begin{align}
\label{eq:vec_conic_a}
\vec{x}^T \myvec{2 & 0\\0 & 0}\vec{x}+ \myvec{-7 & 0}\vec{x}+3=0
\end{align}
If $\myvec{k\\0}$ satisfies \ref{eq:vec_conic_a} then k is the root of the equation \eqref{eq:vec_conic_a}.
\begin{align}
2k^{2}-7k+3 &= 0 \\
\brak{k-3}\brak{2k-1} &= 0
\end{align}
Hence roots are 3 and $\frac{1}{2}$. 
\begin{figure}[!ht]
\centering
\includegraphics[width= \columnwidth]{./figs/conics/parabola1.eps}
\caption{Roots of $2x^2 - 7x + 3 = 0$}
\end{figure}
The python code can be downloaded from
\begin{lstlisting}
codes/conics/parabola1.py
\end{lstlisting}

\item 
$2x^2 + x -4 = 0$ can be expressed as 
\begin{align}
\label{eq:vec_conic_b}
\vec{x}^T \myvec{2 & 0\\0 & 0}\vec{x}+ \myvec{1 & 0}\vec{x} -4=0
\end{align}
If $\myvec{k\\0}$ satisfies \ref{eq:vec_conic_b} then k is the root of the equation \eqref{eq:vec_conic_b}.
\begin{align}
2k^{2}+k-4&=0
\\
\brak{k-1.186}\brak{k+1.686}&=0
\end{align}
Hence roots are 1.186 and 1.686.
\begin{figure}[!ht]
\centering
\includegraphics[width= \columnwidth]{./figs/conics/parabola2.eps}
\caption{Roots of $2x^2 + x -4 = 0$ }
\end{figure}
The python code can be downloaded from
\begin{lstlisting}
codes/conics/parabola2.py
\end{lstlisting}
\item 
$4x^2 + 4\sqrt{3}x + 3 = 0$ can be expressed as 
\begin{align}
\label{eq:vec_conic_c}
\vec{x}^T \myvec{4 & 0\\0 & 0}\vec{x}+ \myvec{4\sqrt{3} & 0}\vec{x}+3=0
\end{align}
If $\myvec{k\\0}$ satisfies \ref{eq:vec_conic_c} then k is the root of the equation \eqref{eq:vec_conic_c}.
\begin{align}
4k^{2}+4\sqrt{3}k+3&=0 \\
\brak{2k+\sqrt{3}}\brak{2k+\sqrt{3}}&=0
\end{align}
Hence both the roots coincide at $\frac{-\sqrt{3}}{2}$.
\begin{figure}[!ht]
\centering
\includegraphics[width= \columnwidth]{./figs/conics/parabola3.eps}
\caption{Roots of $4x^2 + 4\sqrt{3}x + 3 = 0$}
\end{figure}
The python code can be downloaded from
\begin{lstlisting}
codes/conics/parabola3.py
\end{lstlisting}

\item 
$2x^2 +x + 4 = 0$ can be expressed as 
\begin{align}
\label{eq:vec_conic_d}
\vec{x}^T \myvec{2 & 0\\0 & 0}\vec{x}+ \myvec{1 & 0}\vec{x}+4=0
\end{align}
If $\myvec{k\\0}$ satisfies \ref{eq:vec_conic_d} then k is the root of the equation \eqref{eq:vec_conic_d}.
\begin{align} 
2k^{2}+k+4 = 0
\end{align}
The roots are complex and conjugate i.e. $(-0.25 + i1.39)$ and $(-0.25 - i1.39)$
\begin{figure}[!ht]
\centering
\includegraphics[width= \columnwidth]{./figs/conics/parabola4.eps}
\caption{Roots of $2x^2 +x + 4 = 0$ }
\end{figure}
The python code can be downloaded from
\begin{lstlisting}
codes/conics/parabola4.py
\end{lstlisting}
\end{enumerate}

\end{enumerate}